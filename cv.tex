%------------------------------------
% Dario Taraborelli
% Typesetting your academic CV in LaTeX
%
% URL: http://nitens.org/taraborelli/cvtex
% DISCLAIMER: This template is provided for free and without any guarantee 
% that it will correctly compile on your system if you have a non-standard  
% configuration.
% Some rights reserved: http://creativecommons.org/licenses/by-sa/3.0/
%------------------------------------

%!TEX TS-program = xelatex
%!TEX encoding = UTF-8 Unicode

\documentclass[10pt, a4paper]{article}
\usepackage{fontspec} 

% DOCUMENT LAYOUT
\usepackage{geometry} 
\geometry{a4paper, textwidth=5.5in, textheight=8.5in, marginparsep=7pt, marginparwidth=.6in}
\setlength\parindent{0in}

% FONTS
\usepackage[usenames, dvipsnames]{color}
\usepackage[english]{babel}
\usepackage[utf8]{inputenc}
\usepackage{xltxtra}
\defaultfontfeatures{Mapping=tex-text}
\setromanfont [Ligatures={Common}, Numbers={OldStyle}, Variant=01]{Linux 
Libertine O}
\setmonofont[Scale=0.8]{Monaco}

% ---- CUSTOM COMMANDS
\chardef\&="E050
\newcommand{\html}[1]{\href{#1}{\scriptsize\textsc{[html]}}}
\newcommand{\pdf}[1]{\href{#1}{\scriptsize\textsc{[pdf]}}}
\newcommand{\doi}[1]{\href{#1}{\scriptsize\textsc{[doi]}}}
% ---- MARGIN YEARS
\usepackage{marginnote}
\newcommand{\amper{}}{\chardef\amper="E0BD }
\newcommand{\years}[1]{\marginnote{\scriptsize #1}}
\renewcommand*{\raggedleftmarginnote}{}
\setlength{\marginparsep}{7pt}
\reversemarginpar

% HEADINGS
\usepackage{sectsty} 
\usepackage[normalem]{ulem} 
\sectionfont{\mdseries\upshape\Large}
\subsectionfont{\mdseries\scshape\normalsize} 
\subsubsectionfont{\mdseries\upshape\large} 

% PDF SETUP
% ---- FILL IN HERE THE DOC TITLE AND AUTHOR
\usepackage[bookmarks, colorlinks, breaklinks, 
% ---- FILL IN HERE THE TITLE AND AUTHOR
	pdftitle={Simon Bihel - vita},
	pdfauthor={Simon Bihel},
	pdfproducer={http://nitens.org/taraborelli/cvtex}
]{hyperref}  
\hypersetup{linkcolor=blue,citecolor=blue,filecolor=black,urlcolor=MidnightBlue} 

\pagenumbering{gobble}
% DOCUMENT
\begin{document}

{\LARGE Simon Bihel}\\[1cm]
 \\	
\texttt{3} impasse Sainte Julitte\\
lotissement le Clos Saint Cyr\\
\texttt{14400} Vaucelles, Calvados,
France\\[.2cm]
Phone: \texttt{(+33)6 84 57 38 21}\\[.2cm]
Mail: \href{mailto:simon.bihel@ens-rennes.fr}{simon.bihel@ens-rennes.fr}\\ 
Website: \href{http://simonbihel.me}{simonbihel.me}\\
\vfill
 Born on the:  December 5, 1995---Bayeux, France\\
Nationality:  French

\hrule
\section*{Experience}
\noindent
\years{2017}\textsc{Summer Internship}, \textit{Automated Test Data Generation for Dynamically Typed Programming Languages} with Shin Yoo, COINSE Lab, KAIST\\
\years{2016}\textsc{Summer Internship}, \textit{Specifying The Experimental Scenarios For Simulated Cloud Studies} with Martin Quinson \& Anne-Cécile Orgerie, IRISA \& Inria Rennes
\section*{Education}
\noindent
\years{2015\textasciitilde\ 2018}\textsc{Magistère Informatique}
(\textasciitilde\ Master in Computer Science), ENS Rennes \& Rennes 1
University\\
\years{2014\textasciitilde\ 2015}\textsc{1st \& 2nd year Computer Science Bachelor}, highest honors, valedictorian, Caen University\\
\years{2013}\textsc{Science High school diploma}, Computer Science specialty, with high honors


\section*{Certifications}
\years{2014}\textsc{Driving licence} (permis B)\\
% \years{2015}\textsc{C2I Niveau 1}, Caen University\\
% \years{2015}\textsc{MOOC Logique Informatique}, F.U.N. (ENS Cachan)

\section*{Computer skills}
Here are some of my skills, note that I don't have a professional level in either of them.

Python; \LaTeX{}; C\@; Scala; C++; Go; Haskell.

\section*{Foreign languages skills}
\textsc{French}, native speaker;\hfill
\textsc{English}, fluent;\hfill
\textsc{Korean}, elementary proficiency

\section*{Courses projects}
During the second year of Bachelor, I did a project based on the \textit{Faster Than Light} game. It consisted of building an automated fights simulator and then using it to build the most powerful ships.

During the first year of Master, I worked on a project for evaluating a \textit{Processing-in-Memory} architecture using the \textit{k-means} algorithm.


\end{document}
